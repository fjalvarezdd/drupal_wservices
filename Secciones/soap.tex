\section{\Huge{32.2 SOAP en Drupal}}

SOAP (Simple Object Access Protocol) es un protocolo orientado al acceso a objetos, que nos permite intercambiar 
información. Al igual que Al igual que XML-RPC y REST, el protocolo SOAP utiliza XML como formato de 
intercambio. Funciona básicamente como una base para nuestros interfaces de Servicios Web y permite construir 
un entorno de Servicio Web sobre el protocolo. Una de las principales ventajas de SOAP es que éste se encuentra 
de manera nativa a partir de la versión PHP5, con lo que no necesitaremos librarías externas.

SOAP funciona haciendo peticiones RPC (Remote Procedure Call) a un servidor externo. En esta sección veremos 
cómo integrar SOAP en Drupal para hacer peticiones a un WSDL de prueba, como el que provee la W3C para convertir 
métricas de temperatura.

La manera en la que la llamada se ejecuta es codificando el mensaje en formato XML por el cliente SOAP y 
ese mensaje es enviado a través del protocolo HTTP al servidor de aplicaciones Web externo. Para una mayor 
seguridad, se peude utilizar el protocolo HTTPS, por ejemplo, si estamos tratando con un Servicio Web de un banco.
   
Drupal se sirve de dos maneras diferentes a la hora de consumir utilizando SOAP. Drupal Soporta la extensión SOAP  
de PHP 5.x como mencionamos y de la extensión NuSOAP. NuSOAP no es una extensión SOAP sino más bien un conjunto 
de clases PHP que permiten la creación y consumo de Servicios Web. No se necesita instalar ninguna exptensión PHP 
para trabajar con NuSOAP y soporta las especificaciones de SOAP 1.1, pudiendo también generar documentos WSDL.

\section{\Large{32.2.1 Mensaje SOAP}}

SOAP se basa en XML por lo que se considera de lectura humana, pero hay un esquema específico que debe 
ser respetado. Primero vamos a dividir un mensaje SOAP, eliminando datos para estudiar los elementos 
específicos que componen un mensaje SOAP.

\begin{verbatim}
<?xml version="1.0"?>
<soap:Envelope
 xmlns:soap="http://www.w3.org/2001/12/soap-envelope"
 soap:encodingStyle="http://www.w3.org/2001/12/soap-encoding">
 <soap:Header>
  ...
 </soap:Header>
 <soap:Body>
  ...
  <soap:Fault>
   ...
  </soap:Fault>
 </soap:Body>
</soap:Envelope>
\end{verbatim}

Esto podría parecer sólo un archivo XML normal, pero lo que lo convierte en un mensaje SOAP es el elemento 
raíz \textit{Envelope} con el espacio de nombres con http://www.w3.org/2001/12/soap-envelope. El atributo 
\textit{soap: encodingStyle} determina los tipos de datos utilizados en el archivo, pero SOAP en sí no 
tiene una codificación predeterminada.

\textit{soap:Envelope} es obligatorio, pero el elemento siguiente, \textit{soap: Header}, es opcional y 
por lo general contiene información relevante para la autenticación y gestión de sesiones. 
El protocolo SOAP no ofrece ninguna \textit{built-in} de autenticación, pero se lo permite a los desarrolladores 
incluirlo a través de esta etiqueta de cabecera.

A continuación está el elemento requerido \textit{soap:Body} que contiene el mensaje real RPC, incluyendo 
los nombres de método y, en el caso de una respuesta, los valores devueltos del método. El elemento 
\textit{soap:Fault} es opcional, si está presente, se tiene algún mensaje de error o información de estado 
para el mensaje SOAP y debe ser un elemento secundario de \textit{soap:Body}.

Ahora que ya entendemos los fundamentos de lo que constituye un mensaje SOAP, echemos un vistazo a la 
solicitud SOAP y los mensajes de respuesta. Vamos a comenzar con una petición.

\begin{verbatim}
<?xml version="1.0"?>
<soap:Envelope
 xmlns:soap="http://www.w3.org/2001/12/soap-envelope"
 soap:encodingStyle="http://www.w3.org/2001/12/soap-encoding">
 <soap:Body xmlns:m="http://www.yourwebroot.com/stock">
  <m:ObtenerPreciosMercado>
   <m:NombreMercado>IBM</m:NombreMercado>
  </m:ObtenerPreciosMercado>
 </soap:Body>
</soap:Envelope>
\end{verbatim}

El anterior es un ejemplo de mensaje SOAP de solicitud para obtener el precio de las acciones de una 
empresa en particular. Dentro de \textit{soap:Body} se observa el elemento \textit{ObtenerPreciosMercado} que es 
específico de la aplicación. No es un elemento SOAP, y toma su nombre de la función en el servidor que se 
llamará para esta solicitud. \textit{NombreMercado} también es específico de la aplicación y es un argumento 
para la función.

El mensaje de respuesta es similar a la solicitud:

\begin{verbatim}
<?xml version="1.0"?>
<soap:Envelope
 xmlns:soap="http://www.w3.org/2001/12/soap-envelope"
 soap:encodingStyle="http://www.w3.org/2001/12/soap-encoding">
 <soap:Body xmlns:m="http://www.yourwebroot.com/stock">
  <m:ObtenerPreciosMercadoResponse>
   <m:Precio>183.08</m:Precio>
  </m:ObtenerPreciosMercadoResponse>
 </soap:Body>
</soap:Envelope>
\end{verbatim}

En el elemento \textit{soap:Body} hay un elemento \textit{ObtenerPreciosMercadoResponse} con un hijo \textit{Precio}, 
que contiene los datos de retorno. Como es de suponer, tanto \textit{ObtenerPreciosMercadoResponse} tanto \textit{Precio} 
son específicos para esta aplicación.
 
Ahora que ya hemos visto un ejemplo de petición y respuesta y entendemos la estructura de un mensaje SOAP, 
vamos a instalar NuSOAP y construir un cliente SOAP y el servidor para probar la generación de este tipo de 
mensajes.

\subsection{\large{Instalando SOAP en PHP}}

Antes de utilizar SOAP en frameworks de PHP, como Drupal, hay que asegurarse de que la extensión SOAP 
está instalado y habilitada en nuestro servidor. SOAP se ejecuta como una extensión de PHP. Podemos revisar 
el archivo de configuración PHP, o pulse en el link para comprobar su versión de PHP, en su la página de 
informes de Drupal para comprobar la configuración de PHP y ver si SOAP es habilitado. 

En muchos servidores compartidos o dedicados, la extensión debe ser auto-activada, por lo que no tendrás que 
hacer nada especial para que funcione. Sin embargo, en caso de que la extensión no está activada, tendremos que habilitarla.
Podemos abrir el fichero de configuración de PHP (\textit{php.ini}) y buscar el siguiente texto desde en el apartado 
\textit{Configure Command}: \verb|--enable-soap|. Este comando le muestra si SOAP está habilitado. 

A continuación, podemos acceder a la página de información de PHP (utilizando el método \verb|phpinfo()|) y 
buscar la sección específica para la extensión SOAP y deberíamos tener algo como lo siguiente:

\begin{figure}
  \centering
    \includegraphics[width=1\textwidth]{Assets/Soap/Imagenes/phpinfo.png}
  \caption{Extensión SOAP}
\end{figure}

Esto nos muestra que el cliente SOAP y el servidor SOAP están activados en nuestra configuración de PHP. 
Teniendo en cuenta que tenemos el paquete SOAP en nuestra máquina, podemos activarlo desde un Terminal 
de la siguiente manera, teniendo en cuenta que la distribución de Linux es Fedora/CentOS y que el usuario tiene 
permisos para hacerlo (\textit{root}):

\begin{verbatim}
yum install php-soap
\end{verbatim}

Ahora que tenemos nuestro entorno listo, revisaremos un ejemplo de cómo se puede utilizar SOAP en Drupal, 
para solicitar datos a través de una interacción con Servicios Web de un servidor externo. En este caso, 
nos basaremos de un Servicio Web de W3School de conversión, y lo utilizaremos para pasar grados de Celsius a 
Fahrenheit. Es un ejemplo sencillo pero con él aprenderemos cómo crear cualquier peticiones en SOAP y 
podremos extrapolarla a nuestros requerimientos.

La idea de este ejemplo es crear un nuevo tipo de contenido con un campo Celsius, en el cual, cada vez que 
creemos un nodo de este tipo, llame a un Servicio Web SOAP y devuelva la llamada con la temperatura introducida, 
convertida a grados Fahrenheit.

\section{\Large{32.2.2 Uso del módulo Web service client}}

El módulo \textbf{Web service client} es un módulo de la comunidad Drupal que contiene una API que incorporará 
a nuestra instalación de Drupal y que nos permitirá comunicarnos a través de la extensión de PHP 5.x SOAP. Esta 
API nos permitirá además integrarnos con otros Servicios WEB REST, que veremos más adelante. La API se sirve 
del módulo \textbf{Rules}, ya que integra las operaciones de Servicios Web como acciones de \textbf{Rules},  
aunque también puede servirse de otros módulos. 
  
A continuación, mostraremos cómo instalar y configurar el módulo y, finalmente, cómo usarlo para conectarse a 
un Servicio Web externo.

\subsection{\large{Requisitos}}

Para comenzar, veamos primero cómo resolver todos los requisitos para llegar a comunicarnos. Para empezar, crearemos 
el tipo de contenido \textit{Celsius a Fahrenheit}, que nos servirá para almacenar la información en Celsius y 
activar la llamada al Servicio Web para recoger la misma temperatura en Fahrenheit.

Crearemos este nuevo tipo desde \textit{Structure / Content types} y \textit{Add content type}. Lo llamaremos 
``Celsius a Fahrenheit'' y tendrá un campo \textit{Float} como muestra la siguiente pantalla:

\begin{figure}
  \centering
    \includegraphics[width=1\textwidth]{Assets/Soap/Imagenes/content_type.png}
  \caption{Tipo de contenido ``Celsius a Fahrenheit''}
\end{figure}
 
A continuación, instalaremos y configuraremos el módulo \textit{Rules}. Lo descargaremos de su página 
\verb|http://drupal.org/project/rules|, hasta el momento, utilizando la versión \verb|v.7.x-2.2|.

Lo descomprimimos y lo movemos al directorio de módulos de nuestro Drupal (\textit{/sites/all/modules}). 
Ahora nos dirigimos a la página de administración de módulos de Drupal y activamos \textbf{Rules} y \textbf{Rules UI}:

\begin{figure}
  \centering
    \includegraphics[width=1\textwidth]{Assets/Soap/Imagenes/rules_instalacion.png}
  \caption{Instalación de los módulos Rules y Rules UI}
\end{figure}

\subsection{Instalando y configurando Web service client}

Descarge el módulo desde su página: \verb\http://drupal.org/project/wsclient\. En el momento en el que se 
escribieron estas líneas, el módulo se encontraba en la versión \verb|v.7.x-1.0-beta2|. Una vez que se ha descargado 
el paquete, tendremos que:

\begin{itemize}
  \item Descromprimirlo en el directorio \textit{module} de nuestro drupal: \verb|/sites/all/modules|.
  \item Acceder a la página de módulos de nuestro Drupal y activarlo.
  \item Instalaremos los módulos \textbf{Web service client}, \textbf{Web service client SOAP} y \textbf{Web service client UI}
\end{itemize}

A continuación, nos disponemos a crear el descriptor del Servicio Web, utilizando la URL del WSDL del Servicio:

\begin{verbatim}
http://www.w3schools.com/webservices/tempconvert.asmx?WSDL
\end{verbatim}

Para ello, accedemos a \textit{Configuration / Web service descriptions}.
Desde aquí, creamos un nuevo descriptor, el cual llamaremos \textbf{Conversor de Temperatura} e indicaremos que el Servicio es SOAP.

\begin{figure}
  \centering
    \includegraphics[width=1\textwidth]{Assets/Soap/Imagenes/crear_descriptor_servicio.png}
  \caption{Creación de descriptor del Servicio Web}
\end{figure}

Desde la nueva ventana podremos ver los métodos que implementa el servicio a los que podremos acceder desde Drupal, 
en nuestro caso, vamos a utilizar el método \textit{CelsiusToFahrenheit}. También podremos comprobar qué tipos de 
datos contempla el descriptor importado. 

Una vez salvado el formulario, cumplimos todas las dependencias necesarias para implementar nuestra llamada. 
Debido a que definimos que sólamente ibamos a llamar al servicio de conversión una vez que guardaramos contenido del tipo 
\textit{Celsius a Fahrenheit}, nos disponemos a crear una regla que actue de \textit{trigger} una vez que salvemos contenido 
de dicho tipo.

Nuestra llamada al servicio se representaría con el siguiente diagrama:

\begin{figure}
  \centering
    \includegraphics[width=1\textwidth]{Assets/Soap/Imagenes/diagrama_soap.png}
  \caption{Diagrama Servicio}
\end{figure}

Para ello, desde la interfaz de administración (\textit{Configuración / Rules}) añadimos una nueva regla con 
las siguientes características:

\begin{itemize}
  \item Name: \textit{Celsius a Fahrenheit}
  \item Tags: \textit{celsius, fahrenheit}
  \item Reacts on event: \textit{After saving new content}  
\end{itemize}

Tras salvar este formulario, se nos presenta otro nuevo formulario en el que podremos incluir condiciones y 
reacciones de este evento. Tendremos que completarlo con las siguientes condiciones y acciones:

\begin{itemize}
  \item Condition \textbf{content is of given type}:
  
  \begin{itemize}
    \item En el bloque \textit{Conditions} clicamos en \textit{Add condition}.
    \item En el selector, buscamos \textit{Content is of a type} y en el nuevo bloque que nos aparece con los 
    tipos de contenido, elegimos nuestro tipo: \textit{Celsius a Fahrenheit}.
    \item Salvamos el fomulario 
	\end{itemize}
	
  \item Action \textbf{create a data structure}:
  \begin{itemize}
    \item En el bloque \textit{Actions} clicamos en \textit{Add action}.
    \item De este selector, elegimos \textit{Create a data structure} y de la nueva ventana que nos aparece, 
    	el valor será \textit{Conversor de Temperatura: CelsiusToFahrenheit}. Salvamos.
  	\item Elegimos el \textit{Data selector} que sea el campo de nuestro tipo de contenido: \textit{node:field-temperatura}
	\item El último paso será crear una etiqueta para el resultado de nuestra llamada al servicio. En el 
    	 bloque \textit{Provided variables}, nombramos la variable con \textit{Celius} y el nombre de variable, \textit{celsius\_input}.
    \end{itemize}
  \item Action \textbf{call web service}
  \begin{itemize}
    \item Para la llamada al Servicio Web, añadimos una nueva acción \textit{Conversor de Temperatura: CelsiusToFahrenheit}.
    \item Tenemos que tener en cuenta que el dato que utiliza el servicio será \textit{celsius\_input}. 
    \item En este caso, en el apartado \textit{Provider variables} vamos a renombrar el nombre de la variable como 
    \textit{fahrenheit\_output}.
  \end{itemize}
  \item Action \textbf{display result}
  \begin{itemize}
    \item Por último, vamos a elegir que nuestra respuesta se muestre por pantalla con la opción \textit{Show a message on the site}.
    \item El campo que mostraremos será el definido anteriormente como \textit{fahrenheit-output:CelsiusToFahrenheitResult}.
  \end{itemize}
\end{itemize}

Para comprobar cómo funciona nuestra llamada al servicio, ejecutamos la acción, es decir, creamos un nodo del 
tipo \textit{Celsius a Fahrenheit}:

\begin{figure}
  \centering
    \includegraphics[width=1\textwidth]{Assets/Soap/Imagenes/crear_nodo.png}
  \caption{Creando un nodo}
\end{figure}

Tras salvar el nodo anterior obtendríamos la llamada al servicio y su pertinente respuesta, tal como muestra 
el siguiente pantallazo:

\begin{figure}
  \centering
    \includegraphics[width=1\textwidth]{Assets/Soap/Imagenes/resultado_crear_nodo.png}
  \caption{Respuesta del servicio}
\end{figure}




