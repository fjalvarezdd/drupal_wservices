\chapter{Introducción}

Además de como gestor de contenido, Drupal también puede funcionar sirviéndose de contenido externo 
a través de otras aplicaciones Web, tales como Twitter, Facebook o Google. Esta comunicación entre 
Drupal y otras aplicaciones Web es lo que hace del primero una potente plataforma de gestión 
de contenido, tanto propio como externo. Por ejemplo, un desarrollador Drupal puede alimentar su aplicación
con contenido a través de \emph{Aggregation} o \emph{feeds RSS}. 

y otros portales web es lo que hace Drupal una característica rica en la gestión de contenidos
marco capaz de soportar múltiples métodos de alimentación en su contenido
base de datos y estructura del sitio. Por ejemplo, como un desarrollador de Drupal, se puede alimentar
contenido en su sitio con Drupal usando agregación o feeds RSS. El FeedAPI Drupal
(Application Programming Interface) del módulo le permite tomar RSS o XML URL
de los sitios web externos y añadir estos alimentos a su sitio de Drupal. Este es un sólido
método para obtener el contenido de otras aplicaciones web y sitios.
¿Cómo se toma el contenido de todas estas aplicaciones web diferentes y compartir
el contenido de un sitio de Drupal? Esto se está convirtiendo en muy importante en este momento debido a la
gran cantidad de aplicaciones ricas en contenido de gestión que son a la vez en el mercado y
también está disponible en la comunidad de código abierto. Por ejemplo, ¿cómo podemos tomar todas las
de las imágenes que subirlo a nuestro sitio Flickr y compartir esas imágenes con los usuarios de
nuestro sitio de Drupal? En este libro, veremos en detalle en el módulo de Drupal Services, una
módulo contribuido que le ayuda a acelerar las conexiones a los servicios web.
Este módulo nos permitirá integrar a su sitio de Drupal con las aplicaciones externas
las interfaces que utilizan, como XMLRPC, JSON, JSON-RPC, REST, SOAP, y AMF. estos
las interfaces le permitirá a su sitio de Drupal para interactuar y ofrecer los servicios web.
